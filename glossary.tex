\newacronym{bwosg}{BWOSG}{Biuro Współpracy z Otoczeniem Społeczno-Gospodarczym}
\newacronym{cinn}{CINN}{Centrum Innowacji Politechniki Warszawskiej}
\newacronym{cziitt}{CZIiTT}{Centrum Zarządzania Innowacjami i Transferem Technologii}
\newacronym{dbin}{DBIN}{Dział Brokerów Innowacji}
\newacronym{dbms}{DBMS}{(ang. Database Management System) system zarządzania bazą danych}
\newacronym{dowi}{DOWI}{Dział Ochrony Własności Intelektualnej}
\newacronym{inin}{ININ}{Inkubator Innowacyjności Politechniki Warszawskiej}
\newacronym{kpi}{KPI}{(ang. Key Performance Indicator) kluczowy wskaźnik efektywności}
\newacronym{nosql}{NoSQL}{(ang. Not only SQL) rodzaj baz danych nierelacyjnych}
\newacronym{onprem}{on-premises}{infrastruktura lokalna, wewnętrzna (przeciwieństwo rozwiązań chmurowych)}
\newacronym{powerautomate}{Power Automate}{narzędzie Microsoft do automatyzacji przepływów pracy i procesów biznesowych}
\newacronym{powerbi}{Power BI}{(ang. Power Business Intelligence) narzędzie Microsoft do analityki biznesowej i wizualizacji danych}
\newacronym{powerquery}{Power Query}{narzędzie Microsoft do pobierania i transformacji danych}
\newacronym{excel}{Microsoft Excel}{arkusz kalkulacyjny firmy Microsoft służący do przechowywania, organizowania i analizowania danych}
\newacronym{forms}{Microsoft Forms}{narzędzie Microsoft do tworzenia ankiet, kwestionariuszy i formularzy online}
\newacronym{rdbms}{RDBMS}{(ang. Relational Database Management System) Relacyjny System Zarządzania Bazą Danych}
\newacronym{sla}{SLA}{(ang. Service Level Agreement) umowa o gwarantowanym poziomie świadczenia usług}
\newacronym{sql}{SQL}{(ang. Structured Query Language) język zapytań strukturalnych do baz danych}
\newacronym{swot}{SWOT}{(ang. Strengths, Weaknesses, Opportunities, Threats) analiza mocnych i słabych stron oraz szans i zagrożeń}
\newacronym{mlsa}{Microsoft Learn Student Ambassador}{program firmy Microsoft wspierający studentów w rozwijaniu umiejętności technicznych i przywódczych poprzez dostęp do zasobów edukacyjnych, mentoringu i możliwości dzielenia się wiedzą w społeczności akademickiej}
\newacronym{mysql}{MySQL}{system zarządzania relacyjnymi bazami danych o otwartym kodzie źródłowym, wykorzystujący język SQL}
\newacronym{postgresql}{PostgreSQL}{zaawansowany obiektowo-relacyjny system zarządzania bazami danych o otwartym kodzie źródłowym}
\newacronym{sqlserver}{SQL Server}{komercyjny system zarządzania relacyjnymi bazami danych firmy Microsoft}
\newacronym{mongodb}{MongoDB}{nierelacyjna baza danych dokumentowa przechowująca dane w formacie podobnym do JSON, zaprojektowana do obsługi dużych wolumenów danych}
\newacronym{cassandra}{Cassandra}{rozproszony system zarządzania bazami danych typu NoSQL, zaprojektowany do obsługi dużych ilości danych na wielu serwerach}
\newacronym{redis}{Redis}{(ang. Remote Dictionary Server) nierelacyjna baza danych typu klucz-wartość, przechowująca dane w pamięci operacyjnej dla zapewnienia wysokiej wydajności}
\newacronym{neo4j}{Neo4j}{system zarządzania grafowymi bazami danych, specjalizujący się w efektywnym przechowywaniu i przetwarzaniu danych o złożonych relacjach}
\newacronym{it}{IT}{(ang. Information Technology) technologia informacyjna - dziedzina obejmująca systemy komputerowe, oprogramowanie, infrastrukturę i procesy wykorzystywane do przechowywania, przetwarzania i przesyłania informacji}
\newacronym{entraid}{Entra ID}{usługa zarządzania tożsamością i dostępem w chmurze firmy Microsoft (dawniej Azure Active Directory), zapewniająca uwierzytelnianie, autoryzację i zarządzanie dostępem do aplikacji}
\newacronym{microsoft365}{Microsoft 365}{pakiet usług subskrypcyjnych firmy Microsoft, obejmujący aplikacje biurowe (Word, Excel, PowerPoint) oraz usługi chmurowe (OneDrive, SharePoint, Teams), wcześniej znany jako Office 365}
\newacronym{onedrive}{OneDrive}{usługa chmurowa firmy Microsoft służąca do przechowywania, synchronizacji i udostępniania plików między różnymi urządzeniami oraz użytkownikami w ramach ekosystemu Microsoft 365}
\newacronym{mcc}{Microsoft Cost Calculator}{narzędzie firmy Microsoft służące do szacowania kosztów usług i licencji w ekosystemie Microsoft 365, pomagające organizacjom w planowaniu budżetu IT i optymalizacji wydatków na technologię}
\newacronym{roi}{ROI}{(ang. Return on Investment) Zwrot z Inwestycji, wskaźnik finansowy mierzący efektywność inwestycji}
\newacronym{npv}{NPV}{(ang. Net Present Value) wartość bieżąca netto, suma zdyskontowanych przepływów pieniężnych związanych z inwestycją}
\newacronym{pw}{PW}{Politechnika Warszawska}
\newacronym{rodo}{RODO}{rozporządzenie o ochronie danych osobowych (ang. GDPR - General Data Protection Regulation) - unijne rozporządzenie prawne dotyczące ochrony danych osobowych i prywatności obywateli UE, określające zasady gromadzenia, przetwarzania i przechowywania danych osobowych oraz prawa osób, których dane dotyczą}
\newacronym{hipaa}{HIPAA}{(ang. Health Insurance Portability and Accountability Act) to amerykańska ustawa federalna ustanawiająca standardy ochrony danych medycznych pacjentów}
\newacronym{baa}{BAA}{(ang. Business Associate Agreement) to formalna umowa wymagana przez HIPAA, zawierana między podmiotem ochrony zdrowia a jego partnerem biznesowym}
\newacronym{ferpa}{FERPA}{(ang. Family Educational Rights and Privacy Act) amerykańska ustawa federalna chroniąca prywatność danych edukacyjnych uczniów i studentów}
\newacronym{https}{HTTPS}{(ang. Hypertext Transfer Protocol Secure) protokół bezpiecznego przesyłania hipertekstu - rozszerzenie protokołu HTTP z dodatkową warstwą szyfrowania}
\newacronym{sharepoint}{SharePoint}{platforma internetowa firmy Microsoft służąca do zarządzania treścią i dokumentami}
\newacronym{flowengine}{Silnik przepływów}{(ang. Flow Engine) techniczny komponent Microsoft Power Automate, który przetwarza i uruchamia zdefiniowane przez użytkownika automatyzacje, przekształcając zaprojektowane schematy działań w rzeczywiste operacje wykonywane przez system}
\newacronym{api}{API}{(ang. Application Programming Interface) interfejs programowania aplikacji jest to zestaw reguł i protokołów umożliwiających komunikację między różnymi programami i systemami informatycznymi, pozwalający na wymianę danych i funkcjonalności między aplikacjami bez konieczności znajomości ich wewnętrznej implementacji}
\newacronym{etl}{ETL}{(ang. Extract, Transform, Load) proces polegający na: pobieraniu danych z różnych źródeł (ekstrakcja), przekształcaniu ich do odpowiedniego formatu (transformacja) oraz umieszczaniu w docelowej bazie danych (ładowanie)}
\newacronym{mfa}{MFA}{(ang. Multi-Factor Authentication) uwierzytelnianie wieloskładnikowe - metoda zabezpieczania dostępu wymagająca potwierdzenia tożsamości użytkownika za pomocą co najmniej dwóch różnych czynników (np. hasło i kod SMS)}
\newacronym{tls}{TLS}{(ang. Transport Layer Security) protokół bezpieczeństwa sieciowego zapewniający szyfrowanie danych i uwierzytelnianie, następca SSL}
\newacronym{ssl}{SSL}{(ang. Secure Sockets Layer) protokół kryptograficzny do szyfrowania komunikacji sieciowej, poprzednik TLS, obecnie uznawany za przestarzały}