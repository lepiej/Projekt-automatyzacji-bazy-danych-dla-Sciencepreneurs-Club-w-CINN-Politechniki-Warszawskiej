Po zakończeniu etapu programistycznego i pozytywnej ewaluacji jego efektywności, system został poddany serii testów mających na celu zweryfikowanie jego zgodności z założonymi wymaganiami oraz ocenę wydajności w rzeczywistym środowisku pracy \gls{inin}. Przeprowadzono symulacje typowych scenariuszy użytkowania, aby sprawdzić, jak system radzi sobie w warunkach rzeczywistych obciążeń.

\section{Wdrożenie i przekazanie systemu}
Pełne wdrożenie systemu odbyło się w ramach  planu, który obejmował nadanie uprawnień administratorowi oraz przeprowadzenie sesji szkoleniowej. Przygotowano szczegółowe materiały szkoleniowe, w tym instruktażowy film wideo z narracją oraz przewodnik krok po kroku, które stanowiły podstawę do samodzielnego przyswajania wiedzy. Dodatkowo zorganizowano jednorazowe warsztaty, które umożliwiły użytkownikom bezpośrednie zapoznanie się z systemem oraz rozwianie wątpliwości w trakcie praktycznego użycia narzędzia.

Przekazanie systemu użytkownikom jest jednym z najbardziej krytycznych momentów całego procesu wdrożeniowego. To właśnie na tym etapie decyduje się, czy narzędzie zostanie efektywnie zaadaptowane przez organizację, czy też pozostanie jedynie teoretycznym rozwiązaniem. Właściwe przekazanie wiedzy minimalizuje ryzyko błędów, a także redukuje opór przed zmianą, który często towarzyszy wprowadzaniu nowych technologii. W kontekście inkubatora innowacyjności kluczowe było zapewnienie, że nowy system stanie się integralnym elementem procesów zarządzania danymi, a nie jedynie dodatkowym narzędziem wymagającym skomplikowanej obsługi.

Aby zapewnić długoterminową stabilność rozwiązania, opracowano procedury wsparcia technicznego. Po wdrożeniu system był monitorowany przy kolejnej edycji wydarzenia, a administrator procesu mógł zgłosić wszelkie problemy oraz uwagi dotyczące funkcjonalności. 

Długofalowe utrzymanie systemu obejmuje procedury aktualizacji oraz rozwój automatyzacji w ramach istniejącej struktury. System oparty na \gls{excel} jest wystarczający dla wolumenu danych generowanych przez Scientrepreneurs Club. W kontekście długoterminowego wsparcia uwzględniono także konieczność monitorowania aktualizacji produktów Microsoft oraz ich wpływu na funkcjonowanie rozwiązania.

Właściwe wdrożenie i przekazanie systemu użytkownikom to nie tylko zakończenie projektu, ale także początek jego faktycznego funkcjonowania w organizacji. W tym przypadku priorytetem było nie tylko dostarczenie gotowego rozwiązania, ale przede wszystkim zapewnienie, że jego użytkownicy będą potrafili je efektywnie wykorzystać, przyczyniając się do usprawnienia procesów zarządzania danymi w inkubatorze innowacyjności.

\section{Zarządzanie Ryzykiem}
Każdy projekt wdrożeniowy wiąże się z określonymi ryzykami, które mogą wpłynąć na jego powodzenie i długoterminową użyteczność. W przypadku wdrażanego systemu kluczowe zagrożenia można podzielić na ryzyka techniczne, organizacyjne oraz użytkowe. Jednym z najważniejszych wyzwań jest zapewnienie, że system będzie utrzymywany w aktywnym użyciu i dostosowywany do kolejnych edycji programu.

Z perspektywy organizacyjnej ryzykiem jest także brak odpowiedniego przeszkolenia użytkowników i stopniowe porzucenie nowego narzędzia na rzecz dotychczas stosowanych, mniej efektywnych metod. Może to wynikać zarówno z braku jasnych procedur adaptacyjnych, jak i z niewystarczającego wsparcia po wdrożeniu. Dodatkowo, w dłuższej perspektywie istnieje ryzyko utraty kluczowej wiedzy, jeśli nie zostanie odpowiednio udokumentowana oraz przekazana kolejnym użytkownikom systemu.

Aby ograniczyć te zagrożenia, opracowano strategię zarządzania ryzykiem, obejmującą zarówno działania prewencyjne, jak i metody reagowania na ewentualne problemy. Na poziomie technicznym istotne jest wprowadzenie cyklicznych przeglądów i testów kompatybilności systemu z aktualizacjami produktów Microsoft, aby zapewnić jego nieprzerwane działanie. 

W kontekście organizacyjnym kluczowa jest procedura przekazywania wiedzy administratorowi systemu, uwzględniająca szkolenie w formie warsztatu, materiały obsługi systemu oraz dokumentację opisującą kluczowe procesy operacyjne.

Przekazanie projektu użytkownikowi to jeden z najdelikatniejszych momentów wdrożenia. To nie tylko moment technicznego zakończenia prac, ale również strategiczna decyzja dotycząca przyszłości rozwiązania. Jeśli użytkownicy nie zostaną odpowiednio przeszkoleni i nie otrzymają dostatecznego wsparcia, system może zostać porzucony, a inwestycja w jego wdrożenie straci na wartości. Dlatego kluczowe jest nie tylko zapewnienie stabilnego działania systemu, ale również jego integracja z procesami organizacyjnymi w sposób, który naturalnie wpisze go w codzienną pracę \gls{inin}. To właśnie na tym etapie decyduje się, czy system spełni swoją rolę w długoterminowej strategii organizacji, czy stanie się jedynie nieużywanym narzędziem.
