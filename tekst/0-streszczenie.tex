\newpage

%%%%%%%%%%%%%%%%%%%%%%%%%%%%%%%%%%%%
%%	Streszczenie	%%
%%%%%%%%%%%%%%%%%%%%%%%%%%%%%%%%%%%%
{ \fontsize{12}{14} \selectfont
\begin{abstract}
%%%%%%%%%%%%%%%%%%%%%%%%%%%%%%%%%%%%

\begin{center}
Projekt automatyzacji bazy danych dla projektu Sciencepreneurs Club w Centrum Innowacji Politechniki Warszawskiej
\end{center}
Niniejsza praca inżynierska poświęcona jest szczegółowej analizie oraz optymalizacji systemu zarządzania bazą danych członków Sciencepreneurs Club, działającego w ramach Centrum Innowacji Politechniki Warszawskiej. W ramach przeprowadzonej analizy zidentyfikowano kluczowe ograniczenia obecnego systemu, w szczególności związanego z ręcznym przetwarzaniem danych uczestników. Procesy te, obejmujące porównywanie list zgłoszeniowych, kopiowanie informacji kontaktowych, eliminację duplikatów oraz analiz frekwencji, wykazały się wysoką czasochłonnością i podatnością na błędy.

W odpowiedzi na te wyzwania, niniejsza praca inżynierska na podstawie szczegółowej analizy zaproponuje zaprojektowanie i wdrożenie zautomatyzowanego systemu zarządzania danymi, który usprawni kluczowe procesy, poprawi dokładność danych oraz umożliwi skalowanie działań klubu. W pracy szczegółowo omówione zostaną poszczególne komponenty projektu, w tym proces projektowania systemu, jego wdrożenie oraz analiza ekonomiczna oraz użyteczności po jego implementacji. Celem projektu jest stworzenie zautomatyzowanego systemu, który usprawni zarządzanie danymi i pozwoli Sciencepreneurs Club na efektywniejsze wspieranie przedsiębiorczości w społeczności akademickiej.
% koniec streszczenia 

\subsection*{Słowa kluczowe:}
zarządzanie danymi, automatyzacja, usprawnienie procesów, analiza systemu
\end{abstract}}




\newpage
%%%%%%%%%%%%%%%%%%%%%%%%%%%%%%%%%%%%
%% Abstract	%%
%%%%%%%%%%%%%%%%%%%%%%%%%%%%%%%%%%%%
{\selectlanguage{english} \fontsize{12}{14} \selectfont
\begin{abstract}
%%%%%%%%%%%%%%%%%%%%%%%%%%%%%%%%%%%%
\begin{center}
Database automation project for the Sciencepreneurs Club project at the Warsaw University of Technology Innovation Center
\end{center}
This engineering thesis is dedicated to a detailed analysis and optimization of the database management system for members of the Sciencepreneurs Club, operating within the Innovation Center of the Warsaw University of Technology. The conducted analysis identified key limitations of the current system, particularly related to the manual processing of participant data. These processes, which include comparing registration lists, copying contact information, eliminating duplicates, and analyzing attendance, have proven to be highly time-consuming and prone to errors.

In response to these challenges, this engineering thesis will propose, based on a detailed analysis, the design and implementation of an automated data management system that will streamline key processes, improve data accuracy, and enable the scaling of the club's activities. The thesis will thoroughly discuss various components of the project, including the system design process, its implementation, as well as an economic and usability analysis following its deployment. The objective of the project is to create an automated system that enhances data management and enables the Sciencepreneurs Club to more effectively support entrepreneurship within the academic community.
% end of abstract

\subsection*{Keywords:}
data management, automation, process improvement, system analysis


\end{abstract}}