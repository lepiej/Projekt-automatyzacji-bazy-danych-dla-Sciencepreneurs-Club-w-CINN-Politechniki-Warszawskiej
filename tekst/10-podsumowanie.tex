Niniejsza praca inżynierska zaprezentowała kompleksowy projekt automatyzacji procesu zarządzania danymi dla inicjatywy Sciencepreneurs Club w \gls{cinn}. Projekt ten stanowił odpowiedź na zidentyfikowane w toku badań problemy, związane z ręcznym zarządzaniem danymi uczestników wydarzeń, które generowało błędy, powodowało nadmierne obciążenie pracą administracyjną oraz ograniczało możliwości analityczne.
Zaprojektowany system automatyzacji został zrealizowany z wykorzystaniem dostępnych w ramach posiadanych licencji narzędzi \gls{microsoft365}, w szczególności \gls{forms}, \gls{powerautomate}, \gls{excel} oraz \gls{powerquery}. Takie podejście nie tylko umożliwiło optymalizację kosztową przedsięwzięcia, ale również zapewniło łatwą adaptację rozwiązania przez użytkowników znających już podstawy pracy z tymi narzędziami.

Wdrożony system zautomatyzował kluczowe procesy zarządzania danymi, eliminując potrzebę ręcznego kopiowania informacji z formularzy rejestracyjnych do bazy danych. Dzięki wykorzystaniu przepływów pracy w \gls{powerautomate}, dane z formularzy \gls{forms} są automatycznie przenoszone do odpowiednich arkuszy w bazie danych \gls{excel}. Następnie, za pomocą mechanizmów \gls{powerquery}, dane są przetwarzane, aby wykluczyć duplikaty, połączyć informacje z różnych edycji wydarzenia oraz przygotować zbiorcze zestawienia.

Istotnym aspektem projektu było również opracowanie stabilnej architektury systemu, która umożliwia łatwe rozszerzanie funkcjonalności wraz z rosnącą liczbą edycji wydarzeń. Zaprojektowany system został zbudowany zgodnie z zasadą modułowości, co pozwala na dodawanie nowych elementów bez naruszania istniejącej struktury.

\section{Wnioski z realizacji projektu}
Realizacja projektu dostarczyła szeregu wartościowych wniosków, które mogą być wykorzystane zarówno w kontekście opisywanego systemu, jak i w podobnych inicjatywach automatyzacyjnych w przyszłości.

Przede wszystkim, przeprowadzona analiza potwierdziła znaczący potencjał optymalizacyjny w procesach administracyjnych organizacji. Pomimo relatywnie niewielkiej skali projektu Sciencepreneurs Club, ręczna obsługa danych generowała nieproporcjonalnie duże obciążenie operacyjne. Zgodnie z obserwacjami \cite{gontar2019}, tego typu nieefektywności są powszechne w organizacjach, które nie przeszły jeszcze pełnej transformacji cyfrowej.

Drugim istotnym wnioskiem jest dostrzeżenie wartości kompleksowego zmapowania procesu jako fundamentu skutecznej automatyzacji. Jak zauważa \cite{brzezinski2002}, dogłębne zrozumienie procesu przed jego automatyzacją jest kluczowe dla eliminacji nieefektywności systemowych, a nie tylko ich maskowania poprzez technologię. W przypadku opisywanego projektu, szczegółowa analiza istniejącego procesu umożliwiła identyfikację elementów, które można było zautomatyzować, tych wymagających przeprojektowania oraz tych, które powinny pozostać pod kontrolą człowieka.

Wdrożenie projektu potwierdziło również teorię \cite{szmitka2015} o iteracyjnym charakterze transformacji cyfrowej. System nie osiągnął pełnej automatyzacji od razu, jednak znacząco zwiększył poziom zautomatyzowania procesów z praktycznie zerowego do szacowanego na około 50\%. Takie podejście etapowe sprawdziło się jako bardziej adaptacyjne i łatwiejsze do wdrożenia w organizacji.
Co warte podkreślenia, projekt zademonstrował praktyczną wartość podejścia opartego na narzędziach low-code/no-code, które umożliwiają tworzenie rozwiązań automatyzacyjnych bez konieczności posiadania zaawansowanych umiejętności programistycznych. 

\section{Kierunki dalszych prac}
Realizacja projektu automatyzacji zarządzania danymi dla Sciencepreneurs Club stanowi solidny fundament dla dalszego rozwoju systemu, zarówno w kontekście jego funkcjonalności, jak i poziomu automatyzacji. 

Szczegółowe zmapowanie procesu i zdefiniowanie poszczególnych kroków przeprowadzone w ramach niniejszej pracy stanowi idealną podstawę do wprowadzenia dalszej automatyzacji. Jak podkreśla \cite{brzezinski2002}, automatyzacja to proces ciągły, który rzadko osiąga poziom 100\%, jednak każdy wzrost udziału procentowego zautomatyzowanych zadań prowadzi do wymiernych korzyści. W kontekście opisywanego systemu, istnieje potencjał do automatyzacji procesu przygotowania nowej edycji wydarzenia, który obecnie wymaga manualnych interwencji administratora. 
W czasie, który upłynął od pierwszego wdrożenia systemu, Microsoft znacząco rozbudował swoje rozwiązania automatyzacyjne. Aktualnie możliwe jest wykorzystanie bardziej zaawansowanych funkcji Power Automate, które umożliwiają automatyczne tworzenie nowych tabel przed pobraniem i załadowaniem danych. Dodatkowo, część logiki zaimplementowanej w Power Query mogłaby zostać przeniesiona do Power Automate, co uprościłoby architekturę systemu i zwiększyło jego niezawodność. Mechanizmy dostępne w najnowszych wersjach Power Automate, takie jak wyzwalacze czasowe i warunki logiczne, mogłyby zostać wykorzystane do automatycznego generowania raportów oraz powiadomień o zbliżających się wydarzeniach. Implementacja tych funkcji pozwoliłaby na dalsze ograniczenie potrzeby manualnych interwencji administratora, co jest zgodne z koncepcją stopniowej automatyzacji procesów opisaną przez \cite{gupta2023}.

Obecny system koncentruje się głównie na automatyzacji gromadzenia i podstawowego przetwarzania danych. Naturalnym kierunkiem rozwoju byłoby rozszerzenie jego możliwości analitycznych. Wykorzystanie zaawansowanych funkcji Power BI, zintegrowanego z ekosystemem Microsoft 365, umożliwiłoby tworzenie interaktywnych dashboardów prezentujących kluczowe wskaźniki efektywności projektu.
Analiza danych historycznych mogłaby wspierać prognozowanie frekwencji na przyszłych wydarzeniach, co byłoby cennym narzędziem w planowaniu zasobów. 

Opracowany system zarządzania danymi dla Sciencepreneurs Club mógłby zostać zintegrowany z innymi inicjatywami \gls{cinn}. Rozszerzenie go o dodatkowe wydarzenia i projekty pozwoliłoby na budowanie kompleksowej bazy uczestników, co ułatwiłoby koordynację działań promocyjnych i programowych. W perspektywie długoterminowej warto rozważyć migrację systemu do bardziej skalowalnej struktury opartej na relacyjnej bazie danych. Taka strategia byłaby zgodna z kierunkiem rozwoju infrastruktury \gls{it} \gls{cinn}, który zakłada stopniową migrację do rozwiązań chmurowych.

\section{Podsumowanie ryzyka i zarządzania zmianą}

Implementacja systemu wiązała się z koniecznością zarządzania ryzykiem na kilku poziomach. Kluczowym wyzwaniem był potencjalny opór przed zmianą ze strony użytkowników przyzwyczajonych do dotychczasowych metod pracy. Zastosowanie strategii włączającej interesariuszy w proces projektowania oraz kompleksowy program szkoleniowy z materiałami wideo i instrukcjami krok po kroku pomogły zminimalizować to ryzyko.

W wymiarze technicznym, przyjęcie architektury bazującej na \gls{excel} zamiast dedykowanej bazy danych stanowiło kompromis pomiędzy natychmiastową implementacją a optymalną wydajnością. Strategia ta niesie za sobą ryzyko związane z ograniczeniami skalowalności przy znaczącym wzroście wolumenu danych, jednak modułowa konstrukcja systemu umożliwia relatywnie łatwą migrację do bardziej zaawansowanych rozwiązań w przyszłości.

Istotnym zagrożeniem długoterminowym pozostaje kwestia ciągłości wiedzy o systemie – przy zmianie personelu może dojść do utraty know-how dotyczącego architektury i obsługi rozwiązania. Ryzyko to zostało zminimalizowane poprzez szczegółową dokumentację, instrukcje wideo oraz materiały szkoleniowe, które zapewniają zasoby do przekazu wiedzy.

Istnieje również relatywnie niskie ryzyko wynikające z zależności od ekosystemu Microsoft. Chociaż wykorzystanie narzędzi Microsoft 365 znacząco obniżyło koszty implementacji i zwiększyło dostępność rozwiązania, stwarza to jednocześnie zależność od polityki licencyjnej oraz strategii rozwoju produktów Microsoft. Plan zarządzania tym ryzykiem obejmuje regularne monitorowanie zmian w ofercie Microsoft oraz utrzymywanie zdolności do potencjalnej migracji do alternatywnych rozwiązań.

\section{Podsumowanie efektów wdrożenia i efektywności ekonomicznej}
Wdrożony system automatyzacji zarządzania danymi uczestników stanowi przykład udanej transformacji cyfrowej w środowisku akademickim. Szczegółowa analiza efektywności projektu, przedstawiona w rozdziale 7, potwierdziła zarówno operacyjne, jak i ekonomiczne korzyści wynikające z automatyzacji.

Z perspektywy operacyjnej, system znacząco zredukował czas potrzebny na przetwarzanie danych – zadania, które wcześniej zajmowały łącznie około 6,33 godziny na jedno wydarzenie, obecnie wymagają zaledwie 0,67 godziny, co przekłada się na oszczędność 5,67 godziny na każdą edycję wydarzenia. Przy prognozowanej liczbie 10 wydarzeń rocznie oznacza to roczną oszczędność czasu na poziomie 56,7 godzin. Poziom automatyzacji systemu wzrósł z około 12,5\% do 50\%, eliminując większość manualnych, podatnych na błędy operacji.

Przeprowadzona analiza ekonomiczna wykazała, że przy całkowitym koszcie wdrożenia wynoszącym 6 460 zł i rocznych kosztach operacyjnych na poziomie 4 500 zł, system generuje roczne oszczędności finansowe w wysokości 4 820 zł. W perspektywie trzyletniej, pomimo ujemnej wartości \gls{npv} (-5 569 zł), projekt wykazuje dodatni wskaźnik ROI na poziomie 15,5\%, co oznacza, że każda zainwestowana złotówka przynosi około 0,16 zł zysku ponad poniesione nakłady. W kontekście instytucji publicznej, gdzie głównym celem nie jest maksymalizacja zysku, a efektywne realizowanie zadań publicznych, uzyskane wskaźniki efektywności ekonomicznej można uznać za satysfakcjonujące.

Analiza efektywności jakościowej wykazała również znaczącą poprawę satysfakcji administratora systemu – wzrost z oceny 3,5/5 do 4,8/5 w pięciostopniowej skali Likerta. Dokładność przetwarzania danych utrzymała się na poziomie 100\%, co potwierdza niezawodność wdrożonego rozwiązania.

W wymiarze strategicznym, projekt wspiera szersze cele \gls{pw} związane z cyfrową transformacją procesów administracyjnych. Stanowi praktyczną implementację postulatów zawartych w "Strategii rozwoju Politechniki Warszawskiej do roku 2030"\cite{PW2030}, szczególnie w obszarze "Informacji i otoczenia cyfrowego". Jako inicjatywa pilotażowa, dostarcza cennych doświadczeń i dobrych praktyk, które mogą być wykorzystane w analogicznych projektach w innych jednostkach uczelni.

Przeprowadzona w pracy analiza efektywności stanowiła podstawę dla podjęcia decyzji inwestycyjnej. Pomimo umiarkowanych wskaźników finansowych, uzasadnieniem realizacji projektu były istotne korzyści jakościowe i strategiczne, w tym poprawa warunków pracy administratora, eliminacja czasochłonnych zadań manualnych, standaryzacja procesów oraz dostęp do danych w czasie rzeczywistym. Te niematerialne korzyści, choć trudne do bezpośredniej wyceny, stanowią istotną wartość dodaną projektu.

Najistotniejszym efektem wdrożenia jest jednak zmiana w kulturze pracy – przejście od reaktywnego zarządzania danymi do proaktywnego wykorzystania informacji jako zasobu strategicznego. System nie tylko automatyzuje rutynowe zadania, ale również dostarcza narzędzi analitycznych, które wspierają podejmowanie decyzji i strategiczne planowanie działań Sciencepreneurs Club.

\section{Podsumowanie efektów wdrożenia}
Warto podkreślić, że opracowany w ramach niniejszej pracy system został z powodzeniem wdrożony i był wykorzystywany przez kolejne 7 edycji Sciencepreneurs Club. \gls{inin} wyraził zadowolenie z funkcjonalności systemu, wskazując na znaczącą redukcję nakładu pracy administracyjnej oraz poprawę jakości gromadzonych danych.

Trwałość wdrożenia oraz jego pozytywny odbiór przez użytkowników końcowych stanowią potwierdzenie, że przyjęte założenia projektowe i wybrana architektura systemu odpowiadały rzeczywistym potrzebom organizacji. Jednocześnie identyfikacja potencjalnych obszarów dalszego rozwoju świadczy o elastyczności i skalowalności zaprojektowanego rozwiązania.

Podsumowując, przeprowadzony projekt automatyzacji zarządzania danymi dla Sciencepreneurs Club nie tylko dostarczył praktycznych rozwiązań dla konkretnych wyzwań organizacyjnych, ale również zademonstrował wartość systematycznego podejścia do transformacji cyfrowej procesów administracyjnych. Projekt ten wykracza poza standardowe usprawnienie techniczne – reprezentuje krok w kierunku budowania organizacji opartej na danych, w której informacja stanowi kluczowy zasób wspierający realizację misji edukacyjnej i innowacyjnej Politechniki Warszawskiej. Doświadczenia i wnioski płynące z tego projektu mogą stanowić cenny punkt odniesienia dla podobnych inicjatyw automatyzacyjnych w środowisku akademickim i poza nim. 

\section{Perspektywy zastosowania rozwiązania w innych jednostkach organizacyjnych PW}

Zrealizowany projekt automatyzacji bazy danych dla Sciencepreneurs Club, pomimo swojego ukierunkowania na specyficzne potrzeby \gls{inin}, posiada znaczący potencjał do implementacji w innych jednostkach organizacyjnych Politechniki Warszawskiej. Wartość opracowanego rozwiązania wykracza poza ramy pojedynczego projektu, stanowiąc model do naśladowania dla podobnych inicjatyw w strukturze uczelni.

W kontekście potencjalnych obszarów zastosowania, wdrożone rozwiązanie oparte na platformie \gls{microsoft365} może zostać zaadaptowane przez różnorodne jednostki organizacyjne \gls{pw}. Biura kół naukowych mogłyby wykorzystać system do automatyzacji rejestracji uczestników warsztatów, wykładów i konferencji organizowanych na poszczególnych wydziałach. Jednostki administracyjne zyskałyby narzędzie do zarządzania zapisami na szkolenia wewnętrzne i warsztaty kompetencyjne. Biblioteka Główna \gls{pw} mogłaby zaimplementować podobne rozwiązanie do automatyzacji procesu rejestracji na warsztaty biblioteczne czy szkolenia z zakresu korzystania z baz danych naukowych. Również centra badawcze odniosłyby korzyści, usprawniając proces zarządzania uczestnikami seminariów naukowych i wykładów gościnnych.

Aby umożliwić efektywne wykorzystanie opracowanego rozwiązania w innych jednostkach, niezbędne jest stworzenie kompleksowego modelu transferu wiedzy i technologii. Model ten powinien obejmować warsztatowy pakiet wdrożeniowy z zestawem materiałów szkoleniowych, dokumentacją techniczną i przewodnikami dostosowanymi do potrzeb różnych typów użytkowników. Wartościowym elementem byłoby również utworzenie międzywydziałowej społeczności praktyków - administratorów systemów rejestracyjnych, umożliwiającej wymianę doświadczeń i dobrych praktyk. Centralne repozytorium zawierające gotowe szablony formularzy, przepływów automatyzacji i raportów ułatwiłoby adaptację rozwiązania przez inne jednostki. Uzupełnieniem tego ekosystemu mógłby być program mentoringowy zapewniający wsparcie wdrożeniowe przez osoby doświadczone w implementacji podobnych rozwiązań.

Szersze wdrożenie rozwiązania w strukturach uczelni przyniosłoby wymierne korzyści wynikające z efektu skali. Przede wszystkim nastąpiłaby standaryzacja procesów poprzez ujednolicenie sposobu rejestracji i zarządzania danymi uczestników różnych wydarzeń w obrębie uczelni, co zwiększyłoby przejrzystość i efektywność operacyjną. Rozłożenie kosztów utrzymania i rozwoju systemu na większą liczbę jednostek obniżyłoby jednostkowy koszt wdrożenia, przynosząc optymalizację kosztową. Dodatkową wartością byłaby kumulacja wiedzy technicznej i budowanie uczelnianego know-how w zakresie automatyzacji procesów, co mogłoby prowadzić do dalszych innowacji organizacyjnych. Nie bez znaczenia pozostaje również poprawa doświadczeń użytkowników poprzez zapewnienie spójnego, intuicyjnego interfejsu dla uczestników różnorodnych wydarzeń organizowanych przez Politechnikę Warszawską.

Na podstawie doświadczeń z realizacji projektu dla Sciencepreneurs Club, można sformułować szereg rekomendacji dla przyszłych wdrożeń w innych jednostkach organizacyjnych. Przede wszystkim kluczowa jest dokładna analiza specyfiki jednostki i zmapowanie jej procesów organizacyjnych przed rozpoczęciem implementacji. Zaangażowanie przyszłych administratorów w proces projektowania i testowania systemu znacząco zwiększa prawdopodobieństwo jego skutecznej adopcji. 

Podsumowując, opracowane w ramach niniejszej pracy rozwiązanie może stanowić cenną podstawę do szerszej automatyzacji procesów administracyjnych w strukturach Politechniki Warszawskiej. Odpowiednie podejście do rozpowszechnienia wiedzy i technologii pozwoliłoby na uzyskanie znaczących korzyści skali, przyczyniając się do realizacji strategicznych celów uczelni w zakresie cyfryzacji i optymalizacji procesów organizacyjnych.