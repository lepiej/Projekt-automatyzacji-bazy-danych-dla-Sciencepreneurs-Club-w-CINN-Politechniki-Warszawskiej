We współczesnym świecie dane są jak puls organizacji – to one nadają rytm działaniom, pozwalają mierzyć efektywność i podejmować trafne decyzje. Każde przedsięwzięcie, od międzynarodowych korporacji po małe inicjatywy, opiera się na gromadzeniu i analizowaniu informacji. Firmy wykorzystują dane do prognozowania trendów rynkowych, instytucje badawcze do weryfikowania wyników eksperymentów, a organizatorzy wydarzeń do poznania swojej społeczności i dostosowywania oferty. Metoda, jaką zbieramy i przetwarzamy te dane, ma kluczowe znaczenie dla efektywności całego procesu.~\parencite[s. 13]{gontar2019} 

Początkowo ludzie gromadzili informacje w najprostszy możliwy sposób – zapisując je na papierze, w zeszytach, kartotekach. Następnie wprowadzono komputery do miejsc pracy, a wraz z nimi pojawiły się arkusze kalkulacyjne i bazy danych. Jednak wiele procesów pozostawało manualnych: ktoś musiał ręcznie przepisywać nazwiska uczestników wydarzeń, analizować frekwencję, wysyłać przypomnienia i zestawiać raporty. Na małą skalę takie rozwiązania były wystarczające, ale w miarę wzrostu liczby uczestników i wydarzeń każda kolejna linijka oznaczała większe ryzyko błędów, stracony czas i niepotrzebną frustrację organizatorów.

Takie wyzwanie pojawiło się w ramach inicjatywy Sciencepreneurs Club, realizowanej przez \gls{inin}. Jest to przestrzeń, w której przedstawiciele świata nauki i biznesu spotykają się, wymieniają doświadczenia i nawiązują współpracę. Każde wydarzenie organizowane w ramach tego projektu generuje istotne dane – informacje kontaktowe uczestników, adresy e-mail oraz dane dotyczące tego, na jakie wydarzenia się zapisali.

Aby usprawnić proces gromadzenia tych informacji, wykorzystywane są formularze do rejestracji uczestników i zbierania podstawowych danych. Jest to krok w stronę zwiększenia efektywności, eliminujący konieczność ręcznego wprowadzania danych przez organizatorów. Kolejny etap, przetwarzania tych danych, jest już jednak manualny. Dane wciąż wymagają ręcznego przetwarzania w arkuszach kalkulacyjnych, co wiąże się z ryzykiem błędów, powielaniem informacji i dużą czasochłonnością. Z każdym kolejnym wydarzeniem liczba zgromadzonych danych rośnie, a ich ręczna obsługa staje się coraz bardziej uciążliwa.

To właśnie w tym miejscu pojawia się potrzeba dalszej automatyzacji – stworzenia systemu, który nie tylko zbiera dane, ale także je organizuje, analizuje i przekształca w użyteczne informacje. Organizatorzy mogliby skupić się na rozwijaniu inicjatywy, zamiast poświęcać czas na przepisywanie danych. Automatyzacja procesu zbierania i przetwarzania informacji nie jest już luksusem, lecz koniecznością. Pozwala wyeliminować błędy ludzkie, oszczędzić czas i poprawić jakość zgromadzonych danych. Dzięki nowoczesnym narzędziom, takim jak \gls{powerautomate}, \gls{powerquery} czy \gls{forms}, możliwe staje się stworzenie systemu, który sam zadba o gromadzenie i porządkowanie informacji.

\section{Geneza pracy}
Inspiracją do podjęcia tematu niniejszej pracy dyplomowej była moja praktyka w \gls{inin}, który jest częścią \gls{cinn}. Miałam okazję bezpośrednio zaangażować się w realizację projektów wspierających przedsiębiorczość akademicką, w tym w organizację wydarzeń w ramach Sciencepreneurs Club. Wyzwania związane ze skalowaniem tego projektu stały się dla mnie punktem wyjścia do zaprojektowania zautomatyzowanego systemu procesu rejestracji i zarządzania danymi odbiorców. 

Zainspirowana studiami oraz doświadczeniem zdobytym jako \gls{mlsa}, postanowiłam wykorzystać narzędzia, takie jak \gls{excel}, \gls{powerautomate}, \gls{powerquery} i \gls{forms}, aby stworzyć system, który uprościłby gromadzenie, porządkowanie i aktualizowanie danych uczestników. Projekt automatyzacji procesu zarządzania danymi uczestników wydarzeń idealnie wpisuje się w program studiów, który kładzie duży nacisk na wykorzystanie nowoczesnych narzędzi do przetwarzania i analizy danych, a także na optymalizację procesów biznesowych. Dzięki temu mogłam zastosować w praktyce zdobytą wiedzę i umiejętności, które są niezbędne do realizacji projektów związanych z transformacją cyfrową i efektywnym zarządzaniem procesami.
 
\section{Cel pracy oraz przewidywane wyniki}
Celem niniejszej pracy inżynierskiej jest zaprojektowanie i wdrożenie systemu automatyzacji zarządzania danymi uczestników wydarzeń organizowanych w ramach inicjatywy Sciencepreneurs Club na podstawie szczegółowej analizy procesu. Projekt ma na celu optymalizację procesów związanych z gromadzeniem, przechowywaniem oraz przetwarzaniem danych, które są niezbędne do sprawnej organizacji tych wydarzeń. Główne założenia projektu obejmują stworzenie systemu, który pozwoli na automatyczne zbieranie danych z formularzy rejestracyjnych, ich integrację z bazą danych, a także ich łatwe i szybkie aktualizowanie oraz przetwarzanie.

Projekt będzie wykorzystywał narzędzia takie jak \gls{excel}, \gls{powerautomate}, \gls{powerquery} oraz \gls{forms}, które umożliwią automatyzację procesów i eliminację manualnych interwencji. Dzięki temu proces zarządzania danymi stanie się bardziej efektywny, zminimalizuje się ryzyko błędów, a także zwiększy się szybkość reagowania na potrzeby uczestników wydarzeń. Dodatkowo, system ten ma na celu ułatwienie analizy danych dotyczących frekwencji, zaangażowania i preferencji uczestników, co pozwoli na lepsze dostosowanie przyszłych wydarzeń do ich oczekiwań i potrzeb.

\section{Wprowadzenie do układu i struktury pracy}
Niniejsza praca inżynierska została podzielona na dziewięć rozdziałów, które w sposób usystematyzowany prowadzą czytelnika od teoretycznych podstaw zarządzania danymi, poprzez analizę problemu, aż do szczegółowego opisu zaprojektowanego i wdrożonego rozwiązania automatyzacyjnego.

Pierwszy rozdział stanowi wprowadzenie do tematyki pracy. Przedstawiono w nim genezę projektu, zdefiniowano główne cele oraz zarysowano problematykę zarządzania danymi w kontekście wydarzeń organizowanych przez Sciencepreneurs Club.

W rozdziale drugim omówiono teoretyczne aspekty przechowywania i architektury danych. Zaprezentowano różne typy baz danych, ze szczególnym uwzględnieniem \gls{excel} jako narzędzia do budowy prostych systemów bazodanowych. Przedstawiono również możliwości narzędzi takich jak \gls{powerautomate}, \gls{powerquery} i \gls{forms} w kontekście automatyzacji procesów zarządzania danymi. Rozdział zamyka omówienie kluczowych aspektów projektowania architektury rozwiązań informatycznych.

Rozdział trzeci poświęcono analizie środowiska organizacyjnego, w którym funkcjonuje projekt Sciencepreneurs Club. Przybliżono strukturę i zadania \gls{inin} oraz \gls{cinn}, co pozwala lepiej zrozumieć kontekst organizacyjny projektu.

W rozdziale czwartym przeprowadzono szczegółową analizę i diagnozę istniejącego systemu zarządzania danymi w projekcie Sciencepreneurs Club. Omówiono strukturę obecnego systemu, zidentyfikowano jego ograniczenia oraz przeprowadzono analizę \gls{swot}, która stanowiła punkt wyjścia do projektowania usprawnień.

Rozdział piąty przedstawia organizację planu pracy nad projektem automatyzacji. Zdefiniowano w nim podstawowe założenia projektowe, skład zespołu oraz harmonogram działań, co pozwoliło na systematyczne i metodyczne podejście do realizacji przedsięwzięcia.

Rozdział szósty, będący kluczową częścią pracy, zawiera szczegółowy opis procesu projektowania zautomatyzowanego systemu. Przedstawiono w nim wymagania funkcjonalne i niefunkcjonalne, decyzje technologiczne oraz diagramy czynności ilustrujące przepływ danych w systemie. Omówiono również szczegółowo poszczególne komponenty systemu, w tym \gls{forms}, \gls{powerautomate} oraz mechanizmy przetwarzania danych w \gls{powerquery}.

W rozdziale siódmym przeprowadzono analizę efektywności wdrożonego rozwiązania. Zaprezentowano zarówno aspekty ilościowe (czas przetwarzania danych, automatyzacja procesów), jak i ekonomiczne (koszty wdrożenia i utrzymania systemu).

Rozdział ósmy dotyczy procesu wdrożenia systemu oraz zarządzania ryzykiem. Omówiono w nim strategię implementacji rozwiązania, proces przekazania systemu użytkownikom oraz identyfikację potencjalnych zagrożeń wraz z metodami ich minimalizacji.

Ostatni, dziewiąty rozdział zawiera podsumowanie merytoryczne projektu, wnioski z jego realizacji oraz rekomendacje dotyczące potencjalnych kierunków dalszego rozwoju systemu.

Pracę wzbogacono o bibliografię obejmującą zarówno źródła tradycyjne, jak i internetowe, a także o wykaz użytych skrótów i symboli, spis rysunków oraz tabel. Całość stanowi kompleksową dokumentację projektu automatyzacji bazy danych dla Sciencepreneurs Club, prezentując zarówno teoretyczne podstawy, jak i praktyczne aspekty realizacji przedsięwzięcia informatycznego w środowisku akademickim.