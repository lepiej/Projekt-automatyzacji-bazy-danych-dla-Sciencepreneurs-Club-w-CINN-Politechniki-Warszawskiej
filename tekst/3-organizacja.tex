\section{Inkubator Innowacyjności (ININ)}  \gls{inin} działa jako jednostka organizacyjna podległa  \gls{cinn}. Jego historia sięga 2015 roku, kiedy to został powołany do działania jako komórka organizacyjna w równocześnie utworzonym  \gls{cziitt}.\cite{RegulaminININ2015} Przez lata działalności \gls{inin} w strukturze \gls{cziitt} przyczynił się do rozwoju innowacyjnych projektów i komercjalizacji wyników badań naukowych. Jednak w 2023 roku, w wyniku zmian organizacyjnych na Politechnice Warszawskiej, ogólnouczelniana jednostka organizacyjna – \gls{cziitt} – została zlikwidowana.\cite{ZarządzenieCINN2023} Na jej miejsce powołano \gls{cinn}, które od tej pory stało się kluczową jednostką odpowiedzialną za wspieranie przedsiębiorczości akademickiej oraz zarządzanie procesami transferu technologii.

\gls{inin} jest odpowiedzialny za wspieranie rozwoju przedsiębiorczości akademickiej oraz innowacyjnych projektów. Jego głównym celem jest wspieranie inicjatyw biznesowych studentów, absolwentów, pracowników naukowych oraz przedsiębiorców zewnętrznych, dążących do realizacji innowacyjnych projektów technologicznych.

Wspieranie rozwoju innowacyjnych przedsięwzięć jest realizowane poprzez szereg działań, w tym między innymi wsparcie infrastrukturalne i merytoryczne. \gls{inin} udostępnia startupom przestrzeń biurową oraz infrastrukturę techniczną podczas inkubacji od fazy koncepcyjnej do stabilności rynkowej.  Inkubator prowadzi również kompleksowe programy, takie jak preinkubacja, akceleracja, inkubacja i e-inkubacja, oferując wsparcie na każdym etapie rozwoju przedsiębiorstwa. Programy te zapewniają doradztwo prawne, ekonomiczne oraz pomoc w pozyskiwaniu środków finansowych. Beneficjenci mogą uczestniczyć w szkoleniach z zakresu zarządzania, finansowania oraz prowadzenia działalności gospodarczej. Uczestnicy programów tworzonych przez \gls{inin} mają również platformę do nawiązywania współpracy z partnerami biznesowymi, instytucjami badawczymi oraz inwestorami poprzez organizowane przez Inkubator spotkania biznesowe i konferencje, w tym Sciencepreneurs Club. 

Znajduje się na terenie kampusu Politechniki Warszawskiej pod adresem Rektorska 4 w Warszawie, oferując łatwy dostęp do zasobów uczelni oraz bliskość do innych jednostek naukowych i badawczych.

W strukturze Inkubatora Innowacyjności znajdują się specjaliści odpowiedzialni za wsparcie merytoryczne i techniczne beneficjentów programu. Personel \gls{inin} obejmuje kierownika działu, specjalistów ds. badań i inkubatora oraz specjalistów ds. technicznych, którzy są odpowiedzialni za bieżącą współpracę z beneficjentami oraz realizację projektów.

\section{Centrum Innowacji (CINN)}
\gls{cinn} zostało utworzone 1 maja 2023 roku, zastępując zlikwidowane \gls{cziitt}. (Politechnika Warszawska, 2023) Nowa jednostka została powołana w celu integracji działań związanych z transferem technologii, ochroną własności intelektualnej oraz promocją przedsiębiorczości akademickiej.

CINN składa się z czterech kluczowych działów: \gls{dowi}, \gls{dbin}, \gls{bwosg} oraz \gls{inin}.

Każdy z tych działów jest odpowiedzialny za specyficzne aspekty wsparcia innowacji i transferu technologii, zapewniając kompleksowe podejście do rozwoju innowacyjnych projektów. \gls{cinn} odgrywa kluczową rolę w ochronie i komercjalizacji własności intelektualnej, oferując wsparcie w zakresie ochrony patentowej oraz doradztwa w procesie komercjalizacji wynalazków. \gls{cinn} zajmuje się również promocją ofert badawczych i technologicznych \gls{pw}, wspierając komercjalizację wyników badań.
Wspiera tworzenie spin-offów i spin-outów, łącząc naukę z praktyką biznesową, a także koordynuje projekty badawczo-rozwojowe oraz wspiera ich wdrażanie, zarządzając zasobami oraz finansami.

\section{Politechnika Warszawska (PW)}

Politechnika Warszawska jest publiczną uczelnią akademicką działającą na podstawie ustawy z dnia 20 lipca 2018 r. – Prawo o szkolnictwie wyższym i nauce oraz własnego statutu. ~\parencite[s. 7]{statutPW} Uczelnia posiada osobowość prawną, a jej siedzibą jest miasto stołeczne Warszawa. 

Zgodnie ze Strategią Rozwoju \gls{pw} do roku 2030, uczelnia identyfikuje się jako uznana i rozpoznawalna techniczna uczelnia badawcza, będąca atrakcyjnym ośrodkiem naukowo-dydaktycznym w europejskiej przestrzeni badawczej.  Misją Politechniki Warszawskiej jest "kreatywny udział w kształtowaniu przyszłości – poprzez badania, tworzące nową wiedzę i technologie przyszłości oraz poprzez kształtowanie następnych pokoleń" przy jednoczesnym poczuciu społecznej odpowiedzialności. ~\parencite[s. 11]{PW2030} 

Struktura organizacyjna uczelni, określona w Dziale II Statutu \gls{pw}, obejmuje różnorodne jednostki organizacyjne umożliwiające realizację zadań naukowych, dydaktycznych i administracyjnych. Jednostkami organizacyjnymi Politechniki Warszawskiej są podstawowe jednostki organizacyjne (wydziały, kolegia), ogólnouczelniane jednostki organizacyjne o charakterze badawczym, dydaktycznym lub usługowym oraz jednostki organizacyjne administracji. ~\parencite[s. 16]{PW2030} 

Wydziały, jako podstawowe jednostki organizacyjne, są właściwe do organizowania i prowadzenia działalności naukowej w co najmniej jednej dyscyplinie naukowej oraz działalności dydaktycznej w zakresie kierunków studiów. Politechnika Warszawska prowadzi również filię w Płocku, którą kieruje prorektor ds. Filii. ~\parencite[s. 21]{PW2030} Realizując koncepcję Strategicznych Pól Oddziaływania określonych w Strategii 2030, uczelnia koncentruje swoje działania na czterech kluczowych obszarach: fundamenty naukowe jako natura i aparat jej opisu oraz informacja i otoczenie cyfrowe a także zdrowe, zrównoważone środowisko życia. Wraz z zrównoważonym przemysłem, materiałami i procesami wytwarzania.

W ramach swojej struktury \gls{pw} tworzy także centra i jednostki specjalistyczne, takie jak Centrum Innowacji (\gls{cinn}), które integruje działania związane z transferem technologii i innowacji. Jednostki te wspierają realizację strategicznych celów uczelni w zakresie współpracy z otoczeniem gospodarczym i komercjalizacji wyników badań naukowych.