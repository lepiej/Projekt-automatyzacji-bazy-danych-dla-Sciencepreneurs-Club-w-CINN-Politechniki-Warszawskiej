% Podstawowe ustawienia zgodne z wymogami PW
\usepackage[T1]{fontenc}
\usepackage[utf8]{inputenc}
\usepackage{graphicx}
\usepackage{pdfpages}
\usepackage{ifxetex} % konfiguracja jezyka
\usepackage{ifthen} % konfiguracja jezyka
\usepackage{appendix}
\usepackage[hyphens]{url}
\usepackage[hidelinks]{hyperref}
\usepackage{amsmath} % spis wzorow
\usepackage{tocbasic} % For advanced TOC handling
\usepackage{array} 
\usepackage[T1]{fontenc}

% Konfiguracja jezyka
% Dla pdfLaTeX z językiem polskim
\usepackage[MeX]{polski}
\usepackage[english,polish]{babel}
\usepackage{csquotes}
\selectlanguage{polish}
\usepackage[utf8]{inputenc}

% Ustawienia strony - WYMAGANE
\usepackage[inner=30mm,outer=20mm,top=25mm,bottom=25mm]{geometry}
\usepackage{fancyhdr}
\pagestyle{fancy}
\fancyhf{}
\fancyfoot[LE,RO]{\thepage}

% Czcionka bezszeryfowa - ZALECANE
\usepackage{helvet}
\renewcommand{\familydefault}{\sfdefault}

% Interlinia 1.15 - ZALECANE
\usepackage{setspace}
\setstretch{1.15}

% Numeracja na zewnętrznych stronach - WYMAGANE
\renewcommand{\headrulewidth}{0pt}
\fancypagestyle{plain}{
  \fancyhf{}
  \fancyfoot[LE,RO]{\thepage}  % Ta sama numeracja dla stron z rozdziałami
  \renewcommand{\headrulewidth}{0pt}
}

% Formatowanie akapitów, opcja bez wcięcia z odstępem 4 przed akapitem - DO WYBORU
\setlength{\parindent}{0pt}
\setlength{\parskip}{4pt}

% Domyślnie LaTeX stosuje numerację ciągłą w rodziałach- DO WYBORU 

% Bibliografia BG PW
% Bibliografia 1 zgodna z wymogami wydziału - styl numeryczny
\RequirePackage[
    backend=biber,
    style=numeric,         % Styl numeryczny w bibliografii
    citestyle=authoryear,  % Styl cytowania autor-rok w tekście
    sorting=nyt,
    autopunct=true,
    giveninits=true,       % Inicjały zamiast pełnych imion
    isbn=false,            % Bez wyświetlania ISBN
    url=true,              % URL tylko dla źródeł internetowych
    doi=false,             % Bez wyświetlania DOI
    maxbibnames=99,        % Maksymalna liczba autorów (praktycznie wszyscy)
    uniquework=true,       % Ważne dla wykrywania powtarzających się autorów i dat
    uniquelist=false       % Nie zmieniaj liczby autorów dla odróżnienia źródeł
]{biblatex}

% Bibliografia 2 wczytanie pliku z bibliografią
\addbibresource{bibliografia.bib}

% Bibliografia 3 definiowanie stylu wyświetlania bibliografii
\DeclareNameAlias{default}{family-given} % Nazwisko przed imieniem
\DeclareFieldFormat[book,incollection,thesis]{title}{\textit{#1}} % Formatowanie tytułów książek kursywą
% Bibliografia 4 Configure spacing in citations
\renewcommand*{\nameyeardelim}{\addspace}
\renewcommand*{\postnotedelim}{\addcomma\space}

% Bibliografia 4 redefiniujemy \cite na \parencite, żeby odnośniki w nawiasach działały nawet przy \cite{klucz}:
\let\cite\parencite

% Bibliografia 5 Definiujemy filtry na podstawie słowa kluczowego "online"
\defbibfilter{online}{
  keyword=online
}
\defbibfilter{traditional}{
  not keyword=online
}

% Bibliografia 6 Dostosowanie formatu źródeł internetowych
% ZMODYFIKOWANE dla zgodności z BG PW:
\DeclareFieldFormat{url}{Dostępny w: \url{#1}}  % ZMIENIONE: Dodanie "Dostępny w:" przed URL

% Bibliografia 7 Modyfikacja formatu daty dostępu
\DeclareFieldFormat{urldate}{[dostęp: \thefield{urlday}.\thefield{urlmonth}.\thefield{urlyear}]}  % ZMIENIONE: Dodany dwukropek po "dostęp"

% Bibliografia 9  makro do wyświetlania typu nośnika dla online i lokalizacji i wydawy dla artykulu
\newbibmacro*{type+online}{%
  \printtext{[Online]}%
}
\newbibmacro*{date+extrayear}{%
  \iffieldundef{year}
    {}
    {\printtext[parens]{%
       \printfield{year}%
       \printfield{extrayear}}}}

\newbibmacro*{location+publisher}{%
  \printlist{location}%
  \iflistundef{location}
    {\setunit*{\addcomma\space}}
    {\setunit*{\addcolon\space}}%
  \printlist{publisher}%
  \newunit}
  
% Bibliografia 10 driver dla @online - zgodny z wymogami BG PW
\DeclareBibliographyDriver{online}{%
  \usebibmacro{bibindex}%
  \usebibmacro{begentry}%
  \usebibmacro{author/editor+others/translator+others}%
  \setunit{\printdelim{nametitledelim}}\newblock
  \usebibmacro{title}%
  \printtext{[Online]}%
  \newunit\newblock
  \usebibmacro{byauthor}%
  \newunit\newblock
  \usebibmacro{byeditor+others}%
  \newunit\newblock
  \printfield{version}%
  \newunit\newblock
  \printfield{note}%
  \newunit\newblock
  \usebibmacro{publisher+location+date}%
  \newunit\newblock
  \usebibmacro{chapter+pages}%
  \newunit\newblock
  \printfield{pagetotal}%
  \newunit\newblock
  \iffieldundef{urlyear}
    {}
    {\printtext{\printurldate}}%
  \newunit\newblock
  \printfield{url}%
  \newunit\newblock
  \usebibmacro{finentry}}


% Bibliografia 11 driver dla @misc - dla stron internetowych zgodny z BG PW
\DeclareBibliographyDriver{misc}{%
  \usebibmacro{bibindex}%
  \usebibmacro{begentry}%
  \usebibmacro{author/editor+others/translator+others}%
  \setunit{\printdelim{nametitledelim}}\newblock
  \usebibmacro{title}%
  \iffieldundef{howpublished}
    {}
    {\ifboolexpr{test {\iffieldequalstr{howpublished}{online}} or test {\iffieldequalstr{howpublished}{baza danych online}}}
      {\printtext{[\printfield{howpublished}]}}
      {}}%
  \newunit\newblock
  \usebibmacro{byauthor}%
  \newunit\newblock
  \usebibmacro{byeditor+others}%
  \newunit\newblock
  \printfield{version}%
  \newunit\newblock
  \printfield{note}%
  \newunit\newblock
  \usebibmacro{publisher+location+date}%
  \newunit\newblock
  \usebibmacro{chapter+pages}%
  \newunit\newblock
  \printfield{pagetotal}%
  \newunit\newblock
  \iffieldundef{urlyear}
    {}
    {\printtext{\printurldate}}%
  \newunit\newblock
  \printfield{url}%
  \newunit\newblock
  \usebibmacro{finentry}}

  % Bibliografia 11+ Driver dla @article zgodny z wymogami BG PW
\DeclareBibliographyDriver{article}{%
  \usebibmacro{bibindex}%
  \usebibmacro{begentry}%
  % Autor
  \usebibmacro{author/editor+others/translator+others}%
  \setunit{\printdelim{nametitledelim}}\newblock
  % Tytuł artykułu
  \usebibmacro{title}%
  \newunit\newblock
  \usebibmacro{byauthor}%
  \newunit\newblock
  \usebibmacro{byeditor+others}%
  \newunit\newblock
  % Tytuł czasopisma kursywą
  \printfield[emphasized]{journaltitle}%
  % Typ nośnika dla wersji online
  \iffieldundef{url}{}{\printtext{[Online]}}%
  \newunit\newblock
  % Miejsce wydania i wydawca
  \usebibmacro{location+publisher}%
  \newunit\newblock
  % Rok
  \usebibmacro{date}%
  \newunit\newblock
  % Numer/wolumen/wydanie
  \printfield{volume}%
  \printfield{number}%
  \newunit\newblock
  % Zakres stron
  \printfield{pages}%
  \newunit\newblock
  % Data dostępu dla wersji online
  \iffieldundef{urlyear}
    {}
    {\printtext{\printurldate}}%
  \newunit\newblock
  % URL dla wersji online
  \printfield{url}%
  \newunit\newblock
  % Zakończenie wpisu
  \usebibmacro{finentry}}
  
% Bibliografia 12 zmienna do śledzenia duplikatów
\newcommand{\extrayearflag}[1]{%
  \iffieldequalstr{extrayear}{}{}{\def#1{true}}%
}

% Bibliografia 13 Niestandardowy format wyświetlania cytowania - NUMER zamiast literki
\DeclareCiteCommand{\parencite}
  {\bibopenparen\usebibmacro{prenote}}  % Jawny nawias otwierający
  {\usebibmacro{citeindex}%
   \printtext[bibhyperref]{%
     % Drukujemy autora i rok
     \printnames{labelname}%
     \setunit{\nameyeardelim}%
     \printfield{labelyear}%
     % Sprawdzamy czy to duplikat (ma extrayear)
     \iffieldundef{extrayear}
       {}
       {% Jeśli to duplikat, drukujemy numer zamiast literki
        \setunit{\addspace}%
        \printtext[brackets]{\printfield{labelnumber}}}}%
  }
  {\multicitedelim}
  {\usebibmacro{postnote}\bibcloseparen}  % Jawny nawias zamykający
  % Zapewniamy, że \cite również używa nawiasów
\DeclareCiteCommand{\cite}
  {\bibopenparen\usebibmacro{prenote}}
  {\usebibmacro{citeindex}%
    \printnames{labelname}%
    \setunit{\nameyeardelim}%
    \printfield{labelyear}%
    % Sprawdzamy czy to duplikat (ma extrayear)
    \iffieldundef{extrayear}
      {}
      {% Jeśli to duplikat, drukujemy numer zamiast literki
       \setunit{\addspace}%
       \printtext[brackets]{\printfield{labelnumber}}}}
  {\multicitedelim}
  {\usebibmacro{postnote}\bibcloseparen}

% Ukrywamy literki w citowaniach (ważna zmiana!)
\DeclareFieldFormat{extrayear}{}
  
% Bibliografia 14 %  wersja formatu biblografii, bez literek w roku
\renewbibmacro*{author}{%
  \ifboolexpr{
    test \ifuseauthor
    and
    not test {\ifnameundef{author}}
  }
    {\printnames{author}%
     \iffieldundef{authortype}
       {}
       {\usebibmacro{authorstrg}}}
    {\usebibmacro{labeltitle}}%
  \usebibmacro{date+extrayear}}

% Bibliografia 15 dostosowanie dla \cite, który teraz również używa tego samego formatu
\let\cite\parencite

\BiblatexSplitbibDefernumbersWarningOff

% Podpisy rysunków i tabel - Tytuł tabeli umieszczony nad tabelą - justowany do lewej strony, czcionka o kroju bezszeryfowym rozmiar 9 zalecany - Podpis rysunku umieszczony pod rysunkiem - justowany do lewej strony, czcionka o kroju bezszeryfowym rozmiar 9 - ZALECANE
\usepackage{caption}
\captionsetup[table]{font={sf,footnotesize},position=above,justification=raggedright,singlelinecheck=false}
\captionsetup[figure]{font={sf,footnotesize},position=below,justification=raggedright,singlelinecheck=false}

% Formatowanie nagłówków - ZALECANE
\usepackage{titlesec}
\titleformat{\chapter}{\fontsize{14}{16}\bfseries\selectfont}{\thechapter}{1em}{}
\titleformat{\section}{\fontsize{13}{15}\bfseries\selectfont}{\thesection}{1em}{}
\titleformat{\subsection}{\fontsize{12}{14}\bfseries\selectfont}{\thesubsection}{1em}{}

% Przypisy dolne - ZALECANE
\usepackage[bottom,perpage]{footmisc}
\renewcommand{\footnotesize}{\fontsize{9}{10.5}\selectfont}

% Obsługa akronimów i glosariuszy - zgodne z wymogami BG PW
\RequirePackage[acronym,symbols,nopostdot]{glossaries}
\setglossarysection{section}
\makeglossaries
\loadglsentries{glossary}

\newcommand{\acronymstitle}{Wykaz skrótów i symboli}

\newcommand{\acronymslist}{
  \chapter*{\acronymstitle}
  \pagestyle{plain}
  \addcontentsline{toc}{chapter}{\acronymstitle}
  \printglossary[type=\acronymtype,title={}]
  \printglossary[type=symbols,title={}]
}

\renewcommand*{\glsnamefont}[1]{\textbf{#1}}
\renewcommand*{\glossaryheader}{}
\renewcommand*{\glsgroupskip}{}

% Kod
% Kod 1 Kolory dla listingów kodu
\usepackage{xcolor}
\usepackage{listings}

% Kod 2 Definicja języka M (Power Query)
\lstdefinelanguage{PowerQueryM}{
  keywords={let, in, if, then, else, each, try, otherwise, error, and, or, not, 
            as, is, null, function, type, table, record, list, true, false, meta, 
            shared, section, Text, Number, Date, Time, DateTime, Duration, 
            Logical, Binary, Any, none, nullable, optional, Value},
  keywordstyle=\bfseries\color{blue},
  identifierstyle=\color{black},
  sensitive=true,
  comment=[l]{//},
  morecomment=[s]{/*}{*/},
  commentstyle=\color{darkgray},
  stringstyle=\color{orange},
  morestring=[b]",
  morestring=[b]',
  literate=
    {=>}{$\Rightarrow$}{2}
    {...}{$\ldots$}{1}
    {<>}{$\neq$}{1}
    {=}{{{\color{black}=}}}{1}
    {>}{{{\color{black}>}}}{1}
    {<}{{{\color{black}<}}}{1}
}

% Kod 3 Definicja języka JSON
\definecolor{delim}{RGB}{20,105,176}
\definecolor{numb}{RGB}{106, 109, 32}
\definecolor{string}{rgb}{0.64,0.08,0.08}

\lstdefinelanguage{json}{
    numbers=left,
    numberstyle=\small,
    frame=single,
    rulecolor=\color{black},
    showspaces=false,
    showtabs=false,
    breaklines=true,
    postbreak=\raisebox{0ex}[0ex][0ex]{\ensuremath{\color{gray}\hookrightarrow\space}},
    breakatwhitespace=true,
    basicstyle=\ttfamily\small,
    upquote=true,
    morestring=[b]",
    stringstyle=\color{string},
    literate=
     *{0}{{{\color{numb}0}}}{1}
      {1}{{{\color{numb}1}}}{1}
      {2}{{{\color{numb}2}}}{1}
      {3}{{{\color{numb}3}}}{1}
      {4}{{{\color{numb}4}}}{1}
      {5}{{{\color{numb}5}}}{1}
      {6}{{{\color{numb}6}}}{1}
      {7}{{{\color{numb}7}}}{1}
      {8}{{{\color{numb}8}}}{1}
      {9}{{{\color{numb}9}}}{1}
      {\{}{{{\color{delim}{\{}}}}{1}
      {\}}{{{\color{delim}{\}}}}}{1}
      {[}{{{\color{delim}{[}}}}{1}
      {]}{{{\color{delim}{]}}}}{1},
}

% Kod 4 Podstawowe ustawienia dla wszystkich języków
\lstset{
  basicstyle=\ttfamily\small,
  breakatwhitespace=false,
  breaklines=true,
  captionpos=b,
  numbers=left,
  numberstyle=\tiny,
  showspaces=false,
  showstringspaces=false,
  showtabs=false,
  tabsize=2,
  frame=single
}

% Kod 5 formatowanie listy kodow
\renewcommand{\lstlistingname}{Kod}
\renewcommand{\lstlistlistingname}{Spis kodów}

% Załączniki 1 Formatowania nagłówków w załącznikach
\newcommand{\headingformat}[1]{%
  \vspace{0.5cm}
  \noindent\textbf{\large #1}
  \vspace{0.2cm}
}

% URL
% URL 1 Pozwala na łamanie URL-i w bibliografii
\setcounter{biburlnumpenalty}{100}
\setcounter{biburllcpenalty}{100}
\setcounter{biburlucpenalty}{100}
% URL 2 Hyperlinks
\RequirePackage{hyperref}
\hypersetup{
  unicode=true,
  pdftoolbar=true,
  pdfmenubar=true,
  pdffitwindow=false,
  pdfstartview={FitH},
  linktoc=all,
  pdfnewwindow=true,
  colorlinks=true,
  linkcolor=black,
  citecolor=black,
  filecolor=black,
  urlcolor=black,
  breaklinks=true 
}
% URL 3 Modyfikacja formatowania URL, aby umożliwić łamanie długich adresów
% Już zmodyfikowane wyżej w DeclareFieldFormat{url}

% URL 4 Aktywuj znaki specjalne w URL-ach, aby umożliwić łamanie
\def\UrlBreaks{\do\.\do\@\do\\\do\/\do\!\do\_\do\|\do\;\do\>\do\]%
  \do\)\do\,\do\?\do\'\do+\do\=\do\#\do\-}

% URL 5 Ustaw style URL-i
\urlstyle{same}

% Spis wzorow
% Spis wzorow 1 Definicja nowego licznika dla wzorów
\newcounter{equationcounter}[chapter]
\renewcommand{\theequationcounter}{\thechapter.\arabic{equationcounter}}
% Spis wzorow 2 Definicja nowego środowiska dla wzorów
\makeatletter
% Spis wzorow 3 Definicja komendy do opisu wzoru
\newcommand{\equationcaption}[1]{%
  \refstepcounter{equationcounter}%
  \addcontentsline{loe}{equation}{\protect\numberline{\theequationcounter}#1}%
  \par\noindent\small Wzór \theequationcounter: #1\par%
}
% Spis wzorow 4 Tworzenie nowego pliku dla spisu wzorów
\newcommand{\listofequationsname}{Spis wzorów}
\newcommand{\listofequations}{%
  \chapter*{\listofequationsname}%
  \@mkboth{\MakeUppercase\listofequationsname}{\MakeUppercase\listofequationsname}%
  \@starttoc{loe}%
}
% Spis wzorow 5 Dostosowanie formatu wpisu w spisie wzorów
\newcommand*\l@equation[2]{%
  \ifnum \c@tocdepth >\z@
    \addpenalty\@secpenalty
    \addvspace{1.0em \@plus\p@}%
    \setlength\@tempdima{1.5em}%
    \begingroup
      \parindent \z@ \rightskip \@pnumwidth
      \parfillskip -\@pnumwidth
      \leavevmode
      \advance\leftskip\@tempdima
      \hskip -\leftskip
      #1\nobreak\
      \leaders\hbox{$\m@th\mkern \@dotsep mu\hbox{.}\mkern \@dotsep mu$}\hfill\nobreak\hb@xt@\@pnumwidth{\hss #2}\par
    \endgroup
  \fi%
}
% Spis wzorow 6 Definicja typu dokumentu dla spisu wzorów
\@ifundefined{contentsline}{%
  \newcommand{\contentsline}[3]{\csname l@#1\endcsname{#2}{#3}}%
}{}

\@ifundefined{float@listhead}{}{%
  \float@listhead{\listofequationsname}%
}
\makeatother
% Spis wzorow 7 Dodanie spisu wzorów do spisu treści (opcjonalne)
\newcommand{\addlistofequations}{%
  \cleardoublepage
  \phantomsection
  \addcontentsline{toc}{chapter}{\listofequationsname}
  \listofequations
}
